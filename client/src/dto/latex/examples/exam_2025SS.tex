\documentclass[addpoints,12pt]{exam}

\usepackage{xcolor}
\usepackage[lined,linesnumbered,ruled,commentsnumbered]{algorithm2e} % for algorithm
\usepackage{enumitem}
\usepackage{amsopn,amsmath,amssymb,amsfonts}%amsmath embellishments

\printanswers

\pointpoints{ mark}{ marks}

%\pointsinrightmargin
%\bracketedpoints
%\marginpointname{ \points}
%\marksnotpoints


%% solution text
%\SolutionEmphasis{\color{blue}}

% for boxing the correct answer
\ifprintanswers
\newcommand{\mycorrectchoice}[1]{\CorrectChoice \fbox{#1}}
\else
\newcommand{\mycorrectchoice}[1]{\CorrectChoice #1}
\fi

% explanation
\newcommand{\Exp}[1]{\textcolor{blue}{\underline{\emph{Explanation}:} #1}}
% mark
\newcommand{\MK}[1]{\textcolor{red}{(#1 marks)}}
% marking criteria
\newcommand{\MC}[1]{\textcolor{red}{\underline{\emph{Marking Criteria}:} #1}}

\firstpageheader{}{}{COMPSCI 120}
\runningheader{}{}{COMPSCI 120}

\firstpagefooter{}{Page \thepage\ of \numpages}{}
\runningfooter{}{Page \thepage\ of \numpages}{}






\begin{document}

	\begin{center}
		\vspace*{1cm}
		
		\textbf{\fontsize{20}{8}\selectfont THE UNIVERSITY OF AUCKLAND}
		
		\vspace{0.4cm}
		
		\noindent\rule{0.6\linewidth}{0.4pt}
		
		\vspace{0.5cm}
		
		\textbf{\fontsize{12}{8}\selectfont  SUMMER SEMESTER 2025}
		
		\vspace{0.2cm}
		
		\textbf{\fontsize{12}{8}\selectfont  Campus: City}
		
		\noindent\rule{0.6\linewidth}{0.4pt}
				
		\vspace{0.8cm}
		
		\textbf{\fontsize{12}{8}\selectfont COMPUTER SCIENCE}
		
		\vspace{0.8cm}
		
		\textbf{\fontsize{12}{8}\selectfont  Mathematics for Computer Science}
		
		\vspace{0.8cm}
		
		\textbf{\fontsize{12}{8}\selectfont  (Time Allowed: TWO hours) }
	\end{center}
	
	\vspace{0.8cm}
	
	\qquad
	{\fontsize{12}{12}\selectfont
     \textbf{NOTE:}
        \begin{itemize}%[itemindent=5cm]
        \setlength{\leftskip}{40pt}
          \item This test will begin with \fbox{10} minutes of reading time. You may not write anything in this time.
          \item There are \fbox{\numquestions} questions, on pages \fbox{\pageref{first_page}} to \fbox{\pageref{last_page}}. The total number of marks is \fbox{\numpoints}.  Attempt all of the questions; there is no penalty for incorrect answers, and many problems have partial credit!
          \item There is more than enough space for an answer to every problem in this booklet.  If you find yourself running out of space, this may be a sign that you are overthinking things!
          \item Show all working, and place all answers in this booklet.
            \item Pages \fbox{\pageref{blankpage1}} and \fbox{\pageref{blankpage2}} of this test are left blank, to give you additional paper to work on. They will not be marked.
          \item No calculators or notes are allowed.
          \item Best of luck!
        \end{itemize}
     \setlength{\leftskip}{0pt}

     \vfill
     \begin{center}\gradetable[h][questions]\end{center}

\newpage
	
	
\begin{questions}

	\label{first_page}
	
    \question[1] Under which of the following circumstance, we have $f$ grows faster than $g$?
    \vspace{10pt}
    \begin{choices}
        \choice $f(n) > g(n)$ for all natural number $n \in \mathbb{N}$.
        \choice $f(n) = C g(n)$, where $C$ is a constant coefficient, e.g., $C = 3$.
        \mycorrectchoice{$f(n) = n g(n)$}
        \choice $f(n) = - |g(n)|$
    \end{choices}
    \begin{solution}
        C.

        \Exp{
            \begin{align}
                & \lim_{n \to \infty} \frac{|Cg(n)|}{|g(n)|} = |C| \neq \infty \label{eq1}\\
                & \lim_{n \to \infty} \frac{|n g(n)|}{|g(n)|} = \lim_{n \to \infty} |n| =  \infty \nonumber \\
                & \lim_{n \to \infty} \frac{|-|g(n)||}{|g(n)|} = 1 \neq \infty \nonumber
            \end{align}
        where $f(n) = Cg(n)$ for nonnegative $C$, $f$ and $g$ is a case of $f(n) > g(n)$, but \eqref{eq1} shows it does not necessarily results in a faster $f$.
         }

         \MC{Give 1 mark if say waiting for more ciphertext to do frequent analysis.}
    \end{solution}

     \vspace{20pt}

    \question Suppose that an algorithm takes in a set of size $n$ as an input, and enumerate all subsets of the input set. We know that the running time of this algorithm is $2^n-1$ elementary operations.
    \vspace{10pt}
    \begin{parts}
        \part[1] This is a \fillin algorithm.
         \vspace{10pt}
            \begin{choices}
                \choice constant-runtime
                \choice logarithmic-runtime
                \choice polynomial-runtime
                \mycorrectchoice{exponential-runtime}
            \end{choices}
            \begin{solution}
                D.
            \end{solution}

        \vspace{10pt}

        \part[1] If the input set size is doubled to $2n$, what is running time?
         \vspace{10pt}
            \begin{choices}
                \mycorrectchoice{$4^n-1$.}
                \choice $2^n - 2$.
                \choice $4^n$
                \choice $2^{n^2}- 1$
            \end{choices}
            \begin{solution}
                A.

                \Exp{ $2n \rightarrow n$, $2^{2n}-1 = 4^n-1$. }

            \end{solution}
    \end{parts}

    \vspace{20pt}
    
    % copied from 2020 V1
    \question[1] Which of the following statements is true?\\

        \begin{choices}
        	\choice $\displaystyle \lim_{n\to \infty}\dfrac{n^4+3n+2}{n^5+4n+2}=1$
        	\choice $\displaystyle \lim_{n\to \infty}\dfrac{n!+n\log_5(n)+10^{100}}{10000^{n}+n^5+1}=0$
        	\mycorrectchoice{The function $f(n)=\log_2(n) + 2^n$ grows faster than the function $g(n)=n^2$}
        	\choice The function $f(n)=n^5\log_2(n)+7$ grows faster than the function $g(n)=n^6+1.$
        \end{choices}

    
    
    
    \vspace{20pt}
    

    \question Suppose $T$ is a tree that has more than two vertices. Then,
     \vspace{10pt}
        \begin{parts}
            \part[1] there is(are) \fillin subgraph(s) in $T$. %(Recall that a \emph{subgraph} of graph $G$ is a connected graph that is completely contained in $G$.)
            \vspace{10pt}
                \begin{choices}
                    \choice $0$
                    \choice $2$
                    \choice infinite number of
                    \mycorrectchoice{more than one}
                \end{choices}
                \begin{solution}
                    D.

                    \Exp{all nodes are connected in a tree. The whole tree is the only one subgraph}
                \end{solution}

              \vspace{15pt}

            \part[1] we have \fillin between any two vertices in $T$.
             \vspace{10pt}
                \begin{choices}
                    \mycorrectchoice{only one path that is also a walk}
                    \choice only one path, which is not a walk,
                    \choice only one edge
                    \choice at least two paths
                \end{choices}
                \begin{solution}
                    A.

                    \Exp{A tree is connected, i.e., every two nodes are connected by a path, not a walk as repeated vertices indicates cyclic. But, a tree is acyclic.}
                \end{solution}
         \end{parts}

     \vspace{20pt}

    \newpage


    \question For $n \in \mathbb{N}$ being a natural number, evaluate the following limits.
    	 \vspace{10pt}
    	\begin{parts}
    		\part[1] $\lim\limits_{n\to\infty} \frac{1}{3+e^{\frac{1}{n}}}$     		
    		
    		\vspace{10pt}
    		 \begin{oneparchoices}
    			\choice $1$
    			\choice $\frac{1}{3}$
    			\mycorrectchoice{$\frac{1}{4}$}
    			\choice $0$
    		\end{oneparchoices}
    		\begin{solution}
    			C.
    			
    			\Exp{$\lim\limits_{n\to\infty} e^{\frac{1}{n}} = e^{0} = 1$. So, $\lim\limits_{n\to\infty} \frac{1}{3+e^{\frac{1}{n}}} = \frac{1}{4}$}
    		\end{solution}
    		
    		 \vspace{15pt}
    		
    		\part[1] $\lim\limits_{n\to\infty} \frac{n! + 10}{100^{n}+10^{5}}$
    		
    		\vspace{10pt}
    		\begin{oneparchoices}
    			\choice $\frac{1}{10^5}$
    			\choice $10$
    			\choice $0$
    			\mycorrectchoice{$\infty$}
    		\end{oneparchoices}
    		\begin{solution}
    			D.
    			
    			\Exp{$\text{constant} \ll \text{exponential} \ll \text{factorial}$}
    		\end{solution}
    		
    		\vspace{15pt}
    		
    		
    		\part[1] $\lim\limits_{n\to\infty} \frac{2}{e^{-n}+1}$
    		
    		\vspace{10pt}
    		\begin{oneparchoices}
    			\choice $1$
    			\mycorrectchoice{$2$}
    			\choice $\infty$
    			\choice $0$    			
    		\end{oneparchoices}
    		\begin{solution}
    			B.
    			
    		\Exp{$\lim\limits_{n\to\infty} e^{-n} = 0$. So, $\lim\limits_{n\to\infty} \frac{2}{e^{-n}+1} = 2$}
    		\end{solution}
    	\end{parts}
    	
    	\vspace{20pt}
 
 % copied from 2020 V1   
 \question[1]Which of the following statements about graphs is true?
	\begin{choices}
		\choice Every graph on $n$ vertices with $n-1$ edges is connected.
        \mycorrectchoice{If there is a path from $x$ to $y$, then there must also be a walk from $x$ to $y$.}
		\choice There is a graph on 10 vertices such that 2 vertices have degree 3, 4 vertices have degree 2, 1 vertex has degree 4, and 3 vertices have degree 5.
		\choice There is a graph on 10 vertices such that exactly 2 vertices have degree 1, 4 vertices have degree 3, and 1 vertex has degree 10.
	\end{choices}

 \vspace{20pt}

 \question[1] Consider the claim below.
 		\begin{center} 			
 			\fbox{\textbf{Claim:} For all $x>0$ and $y>0$, $(\frac{\sqrt{x} + \sqrt{y}}{2})^2 \le \frac{x+y}{2}$. }
 		\end{center} 		
 		%What is the correct way to negate this claim.
 		Which of the following assumptions would be the best to use when setting up a contradiction?
 		            \vspace{10pt}
 		\begin{choices}
 			\mycorrectchoice{There exists a pair of $x>0$ and $y>0$ such that $(\frac{\sqrt{x} + \sqrt{y}}{2})^2 > \frac{x+y}{2}$.}
 			\choice For all $x>0$ and $y>0$, $(\frac{\sqrt{x} + \sqrt{y}}{2})^2 > \frac{x+y}{2}$.
 			\choice There exist a pair of $x>0$ and $y\leq0$ such that $(\frac{\sqrt{x} + \sqrt{y}}{2})^2 \leq \frac{x+y}{2}$.
 			\choice There exist a pair of $x\leq0$ and $y\leq0$ such that $(\frac{\sqrt{x} + \sqrt{y}}{2})^2 \leq \frac{x+y}{2}$
 		\end{choices}
 	
           \vspace{20pt}

 \question[1]  Suppose you are given the task to prove the claim below
		 \begin{center} 			
		 	 \fbox{\textbf{Claim:} $\max \{x+a, 0\} =\max \{x,-a\} + a$ for all real numbers $a$ and $x$. }
		 \end{center}
		 Which of the following proof methods is best suited to this task?
		
		 \begin{choices}
		 	\choice Proof by induction.
		 	\mycorrectchoice{Proof by cases}
		 	\choice Proof by construction.
		 	\choice Direct proofs.
		 \end{choices}

\vspace{20pt}

 \question[1] Which of the following is most suited to being proved by construction?
        \begin{choices}
		 	\choice $\sqrt{3}$ is irrational. 
            \choice The sum of the first $n$ natural numbers is $\frac{n(n-1)}{2}$. 
		 	\mycorrectchoice{There exists a connected graph where the degree of every vertex is at least $2$.}
            \choice If $p>2$ is a prime number, then $p+1$ is composite.
		 \end{choices}

\vspace{20pt}

 \question Suppose that you are proving the statement $P(n)$ below by induction. 

      \begin{center}
      		\fbox{$P(n)$ = ``$9^n-1$ is divisible by $8$ for all natural number $n \in \mathbb{N}$".}
      \end{center}
      
      \vspace{10pt}
      
      \begin{parts}
          \part[1] What should you prove in the base case?

          \begin{choices}
            \mycorrectchoice{$9^1-1$ is divisible by $8$.}
            \choice $9^0-1$ is divisible by $8$.
            \choice $9^{n}-1$ is divisible by $8$ for all $n \in \mathbb{N}$.
            \choice $9^{n-1}-1$ is divisible by $8$ for all $n \in \mathbb{N}$.
          \end{choices}
          
          \vspace{10pt}
          
          \part[1] What should you do in the induction step?
          
          \begin{choices}
            \choice Prove both $9^{n} - 1$ and $9^{n+1} - 1$ are divisible by $8$.
            \choice Enumerate $n=2,3,\dotsc$ and prove $9^{n} - 1$ is divisible by $8$.
            \choice Prove $9^{n} - 1$ is divisible by $8$ by assuming $9^{n+1} - 1$ is divisible by $8$.
            \mycorrectchoice{Prove $9^{n+1} - 1$ is divisible by $8$ by assuming $9^{n} - 1$ is divisible by $8$. }
          \end{choices}
      
      \end{parts}

 \question Consider the proposition below.

    	\begin{center}
    		\fbox{\textbf{Proposition:} If $a,b > 0$, then $\frac{ax + by}{a+b} \leq \max \{x,y\}$ for all real valued $x$ and $y$. }
    	\end{center}
    
        \vspace{10pt}
    
        \begin{parts}
          \part[1] Suppose you are proving this proposition by cases. What are all possible cases you need to consider? 
          
          \begin{choices}
            \choice $x > 0$ and $y<0$. 
            \choice $x \ge a$ and $y<b$. 
            \mycorrectchoice{$x \geq y$ and $x<y$.}
            \choice $|x| \ge a$ and $|y|<b$. 
          \end{choices}
          
          \vspace{10pt}
          
          \part[1] Suppose you are proving this proposition by contradiction. What should you do in the proof? 
          
          \begin{choices}
            \choice Assume $a,b\leq 0$ and prove that this contradicts $\frac{ax + by}{a+b} \leq \max \{x,y\}$. 
            \choice Assume there does not exist any $x,y$ such that $\frac{ax + by}{a+b} \leq \max \{x,y\}$ and prove that this contradicts $a,b>0$.
            \mycorrectchoice{Assume there exist $x,y$ such that $\frac{ax + by}{a+b} > \max \{x,y\}$} \\ \fbox{and prove that this contradicts $a,b>0$.}
            \choice Assume there exist $a,b \leq 0$ such that $\frac{ax + by}{a+b} \leq \max \{x,y\}$ and prove that this contradicts $x$ and $y$ are real numbers.
          \end{choices}
          
        \end{parts}  
    	



\newpage

    \question For any natural number $n \in \mathbb{N}$, the factorial is $n! = 1 \cdot 2 \cdot \dotsc \cdot (n-1) \cdot n$.
    \vspace{10pt}
    	\begin{parts}
    		\part[3] Show that $n!$ can be computed by a recursion.
    		\begin{solution}
    				\begin{align}
    					 n!
    					 & = 1 \cdot 2 \cdot \dotsc \cdot (n-1) \cdot n \label{eq1}\\
    					 & =  1! \cdot 2 \cdot \dotsc \cdot (n-1) \cdot  \\
    					 & = 2! \cdot \dotsc \cdot (n-1) \cdot n\\
    					 & = ...... \label{eq2}\\
    					 & = (n-1)! \cdot n \label{eq3}
    				\end{align}
    				Therefore, $n!$ can be done by recursion.
    				
    				\MC{1.5 mark for showing equation \eqref{eq3}; 1.5 mark for all steps from \eqref{eq1} to \eqref{eq2}.}
    	    \end{solution}
    		
    		\vspace{10pt}
    		
    		\part[3] Write a recursive algorithm $\texttt{factorial}(n)$ that computes $n!$ for any input $n$.
    		\begin{solution}
    			
    			\begin{algorithm}[H]
    				\caption{\texttt{factorial}}
    				\SetAlgoLined
    				\SetKwInOut{Input}{input}\SetKwInOut{Output}{output}
    				\SetKwFor{For}{for}{do}{endfor}
    				\SetKwRepeat{Repeat}{repeat}{until}
    				\SetKwIF{If}{ElseIf}{Else}{if}{then}{else if}{else}{endif}
    				\BlankLine
    				\Input{A natural number $n \in \mathbb{N}$.}
    				\Output{$n$}
    				\BlankLine
    				\lIf{n =1}{output $1$}
    				\lElse{output $\texttt{factorial}(n-1)$}
    			\end{algorithm}
    			
    			\MC{1.5 mark for line 1; 1.5 mark for line 2.}
    		\end{solution}
    		
    		\part[2] What is the running time of this algorithm? Explain your answer.
    		\begin{solution}
    			 $n!$
    			
    			\MC{1 mark for correct answer; 1 mark for explanation, e.g., counting the number of multiplications.}
    		\end{solution}
    		
    	\end{parts}	

    \label{last_page}
 
     \newpage

       \topskip0pt
    \vspace*{\fill}
    \begin{center}
    	%\scriptsize
    	\emph{\textcolor{gray}{This page is intentionally left blank.}} \label{blankpage1}
    \end{center}
    \vspace*{\fill}
    
    \newpage
       \topskip0pt
   \vspace*{\fill}
    \begin{center}
    	%\scriptsize
    	\emph{\textcolor{gray}{This page is intentionally left blank.}} \label{blankpage2}
    \end{center}
     \vspace*{\fill}

 %   \question Prove by induction that the statement $P(n)$ below is true for all natural number  $n \in \mathbb{N}$.
%
%      \begin{center}
%      		\fbox{$P(n)$ = ``$9^n-1$ is divisible by $8$ for all natural number $n$".}
%      \end{center}
%
%      \begin{solution}
%      	
%      		\textbf{Base Case: } for $n=1$,
%      		$$ (9^1-1) \% 8 = 8\%8 = 0$$
%
%      		\textbf{Inductive Step: } assume $9^n-1$ for some $n$. Then,
%      		\begin{align*}
%      			(9^{n+1}-1) \% 8
%      			&= 	(9 \cdot 9^n-1) \% 8 \\
%      			&=  (9 \cdot (9^n -1 + 1) -1) \% 8 \\
%      			&=  (9 \cdot (9^n -1 ) +8) \% 8 \\
%      			&= 9 \cdot (9^n -1 ) \% 8 + 8\%8\\
%      			&= 0
%      		\end{align*}
%			Q.E.D.
%      		
%      		\MC{1 mark for base case; 3 marks for inductive step}
%      \end{solution}
%
%    \question Prove the proposition below by contradiction.
%
%    	\begin{center}
%    		\fbox{\textbf{Proposition:} If $a,b > 0$, then $\frac{ax + by}{a+b} \leq \max \{x,y\}$ for all real valued $x$ and $y$. }
%    	\end{center}
%    	Clearly state the assumption when setting up the contradiction.
%    	
%    	 \begin{solution}
%    		
%    		\textbf{Negation: } Assume that for $a,b>0$, $\exists x, y$ such that $\frac{ax + by}{a+b} > \max \{x,y\}$.
%    		
%    		Then,     		
%    		\begin{align}
%    			\frac{ax + by}{a+b} &> \max \{x,y\}  \nonumber \\
%    			& \Longrightarrow ax + by > (a+b) \max\{x, y\}  \nonumber  \\
%    			& \Longrightarrow  a(x- \max\{x, y\}) + b (y-\max\{x, y\}) > 0 \label{eq4}
%    		\end{align}
%    		But, we know that $x- \max\{x, y\} \leq 0$ and $y- \max\{x, y\} \leq 0$. The only way to hold the inequality~\eqref{eq4}  		
%    		is to have at least $a$ or $b$ negative. This contradicts the fact the $a,b>0$. Therefore, we must have $\frac{ax + by}{a+b} \leq \max \{x,y\}$.
%    		
%    		Q.E.D.
%    		
%    		\MC{2 marks for negation; 3 marks for proof.}
%    	\end{solution}
    	
		
    

\end{questions}
	

	
%	\vfill
%	
%\begin{center}\gradetable[h][questions]\end{center}
	
	
\end{document} 