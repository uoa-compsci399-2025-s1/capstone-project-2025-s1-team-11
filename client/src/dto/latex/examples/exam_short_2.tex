\documentclass[addpoints,12pt]{exam}

\usepackage{xcolor}
\usepackage{amsmath,amssymb,amsfonts}
\usepackage{enumitem}
\usepackage{algorithm2e}

\printanswers

\pointpoints{mark}{marks}

% for boxing the correct answer
\ifprintanswers
\newcommand{\mycorrectchoice}[1]{\CorrectChoice \fbox{#1}}
\else
\newcommand{\mycorrectchoice}[1]{\CorrectChoice #1}
\fi

% explanation
\newcommand{\Exp}[1]{\textcolor{blue}{\underline{\emph{Explanation}:} #1}}
% marking criteria
\newcommand{\MC}[1]{\textcolor{red}{\underline{\emph{Marking Criteria}:} #1}}

\firstpageheader{}{}{COMPSCI 230}
\runningheader{}{}{COMPSCI 230}

\firstpagefooter{}{Page \thepage\ of \numpages}{}
\runningfooter{}{Page \thepage\ of \numpages}{}

\begin{document}

\begin{center}
  \vspace*{1cm}
  
  \textbf{\fontsize{20}{8}\selectfont MIDTERM EXAMINATION}
  
  \vspace{0.4cm}
  
  \noindent\rule{0.6\linewidth}{0.4pt}
  
  \vspace{0.5cm}
  
  \textbf{\fontsize{12}{8}\selectfont SPRING SEMESTER 2024}
  
  \vspace{0.2cm}
  
  \textbf{\fontsize{12}{8}\selectfont Campus: West}
  
  \noindent\rule{0.6\linewidth}{0.4pt}
      
  \vspace{0.8cm}
  
  \textbf{\fontsize{12}{8}\selectfont DEPARTMENT OF COMPUTER SCIENCE}
  
  \vspace{0.8cm}
  
  \textbf{\fontsize{12}{8}\selectfont Data Structures and Algorithms}
  
  \vspace{0.8cm}
  
  \textbf{\fontsize{12}{8}\selectfont (Time Allowed: 90 minutes)}
\end{center}

\vspace{0.8cm}

\qquad
{\fontsize{12}{12}\selectfont
   \textbf{INSTRUCTIONS:}
      \begin{itemize}
      \setlength{\leftskip}{40pt}
        \item This exam contains \fbox{\numquestions} questions.
        \item The total number of marks is \fbox{\numpoints}.
        \item Answer all questions.
        \item No electronic devices allowed.
      \end{itemize}
   \setlength{\leftskip}{0pt}
}

\vfill
\begin{center}\gradetable[h][questions]\end{center}

\newpage

\begin{questions}
\label{first_page}

\question[2] What is the time complexity of the following algorithm?

\begin{verbatim}
function mystery(n):
    if n <= 1:
        return 1
    return mystery(n/2) + 1
\end{verbatim}

\begin{choices}
  \choice $O(n)$
  \mycorrectchoice{$O(\log n)$}
  \choice $O(n \log n)$
  \choice $O(n^2)$
\end{choices}

\begin{solution}
  B.
  
  \Exp{
    The function divides n by 2 in each recursive call until n becomes less than or equal to 1.
    Starting from n, the number of times we can divide by 2 until reaching 1 is $\log_2(n)$.
    Therefore, the time complexity is $O(\log n)$.
  }
  
  \MC{
    2 marks for correctly identifying the logarithmic time complexity.
  }
\end{solution}

\vspace{20pt}

\question[2] Which of the following data structures would be most efficient for implementing a priority queue?

\begin{choices}
  \choice Linked List
  \choice Stack
  \mycorrectchoice{Heap}
  \choice Hash Table
\end{choices}

\begin{solution}
  C.
  
  \Exp{
    A heap data structure provides $O(\log n)$ time complexity for insertion and extraction of the minimum/maximum element,
    which are the primary operations in a priority queue. Other data structures like linked lists would require $O(n)$ time
    for finding the minimum/maximum element or maintaining sorted order.
  }
  
  \MC{
    2 marks for correctly identifying a heap as the most efficient structure for a priority queue.
  }
\end{solution}

\vspace{20pt}

\question[2] What is the worst-case time complexity of quicksort?

\begin{choices}
  \choice $O(n)$
  \choice $O(n \log n)$
  \mycorrectchoice{$O(n^2)$}
  \choice $O(2^n)$
\end{choices}

\begin{solution}
  C.
  
  \Exp{
    Quicksort has a worst-case time complexity of $O(n^2)$ when the pivot selection consistently results
    in highly unbalanced partitions, such as when the array is already sorted and the pivot is always
    chosen as the first or last element.
  }
  
  \MC{
    2 marks for correctly identifying the worst-case time complexity.
  }
\end{solution}

\vspace{20pt}

\question Consider the following binary search tree:

\begin{center}
\begin{verbatim}
      10
     /  \
    5    15
   / \     \
  3   7     20
\end{verbatim}
\end{center}

\begin{parts}
  \part[1] What is the height of this tree?
  \begin{choices}
    \choice 4
    \mycorrectchoice{2}
    \choice 3
    \choice 5
  \end{choices}
  \begin{solution}
    B.
    
    \Exp{
      The height of a tree is the number of edges in the longest path from the root to a leaf.
      The longest path in this tree is from 10 to either 3 or 7, which contains 2 edges.
      Thus, the height is 2.
    }
  \end{solution}
  
  \part[1] What would be the result of an in-order traversal of this tree?
  \begin{choices}
    \choice 10, 5, 3, 7, 15, 20
    \choice 3, 5, 7, 10, 20, 15
    \mycorrectchoice{3, 5, 7, 10, 15, 20}
    \choice 3, 7, 5, 20, 15, 10
  \end{choices}
  \begin{solution}
    C.
    
    \Exp{
      In-order traversal visits nodes in the order: left subtree, root, right subtree.
      For this tree: 3, 5, 7, 10, 15, 20.
    }
  \end{solution}
\end{parts}

\vspace{20pt}

\question[2] What is the space complexity of the following recursive function?

\begin{verbatim}
function fibonacci(n):
    if n <= 1:
        return n
    return fibonacci(n-1) + fibonacci(n-2)
\end{verbatim}

\begin{choices}
  \choice $O(1)$
  \choice $O(\log n)$
  \choice $O(n)$
  \mycorrectchoice{$O(2^n)$}
\end{choices}

\begin{solution}
  D.
  
  \Exp{
    This recursive implementation of fibonacci creates a binary tree of recursive calls,
    where each call branches into two more calls. The depth of this recursion tree is approximately n,
    and the number of calls is approximately $2^n$. Since each call uses a constant amount of stack space,
    the total space complexity is $O(2^n)$.
  }
  
  \MC{
    2 marks for identifying the exponential space complexity.
  }
\end{solution}

\label{last_page}
\end{questions}

\end{document} 