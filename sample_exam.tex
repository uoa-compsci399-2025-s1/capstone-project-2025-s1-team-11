\documentclass[addpoints,12pt]{exam}

\usepackage{xcolor}
\usepackage[lined,linesnumbered,ruled,commentsnumbered]{algorithm2e} % for algorithm
\usepackage{enumitem}
\usepackage{amsopn,amsmath,amssymb,amsfonts}%amsmath embellishments

\printanswers

\pointpoints{ mark}{ marks}

% for boxing the correct answer
\ifprintanswers
\newcommand{\mycorrectchoice}[1]{\CorrectChoice \fbox{#1}}
\else
\newcommand{\mycorrectchoice}[1]{\CorrectChoice #1}
\fi

% explanation
\newcommand{\Exp}[1]{\textcolor{blue}{\underline{\emph{Explanation}:} #1}}
% mark
\newcommand{\MK}[1]{\textcolor{red}{(#1 marks)}}
% marking criteria
\newcommand{\MC}[1]{\textcolor{red}{\underline{\emph{Marking Criteria}:} #1}}

\firstpageheader{}{}{COMPSCI 120}
\runningheader{}{}{COMPSCI 120}

\firstpagefooter{}{Page \thepage\ of \numpages}{}
\runningfooter{}{Page \thepage\ of \numpages}{}


\begin{document}

\begin{center}
    \vspace*{1cm}
    
    \textbf{\fontsize{20}{8}\selectfont THE UNIVERSITY OF AUCKLAND}
    
    \vspace{0.4cm}
    
    \noindent\rule{0.6\linewidth}{0.4pt}
    
    \vspace{0.5cm}
    
    \textbf{\fontsize{12}{8}\selectfont  SUMMER SEMESTER 2025}
    
    \vspace{0.2cm}
    
    \textbf{\fontsize{12}{8}\selectfont  Campus: City}
    
    \noindent\rule{0.6\linewidth}{0.4pt}
            
    \vspace{0.8cm}
    
    \textbf{\fontsize{12}{8}\selectfont COMPUTER SCIENCE}
    
    \vspace{0.8cm}
    
    \textbf{\fontsize{12}{8}\selectfont  Sample Exam}
    
    \vspace{0.8cm}
    
    \textbf{\fontsize{12}{8}\selectfont  (Time Allowed: TWO hours) }
\end{center}

\vspace{0.8cm}

\qquad
{\fontsize{12}{12}\selectfont
 \textbf{NOTE:}
    \begin{itemize}
    \setlength{\leftskip}{40pt}
      \item This test will begin with \fbox{10} minutes of reading time. You may not write anything in this time.
      \item There are \fbox{\numquestions} questions. The total number of marks is \fbox{\numpoints}.
      \item Show all working, and place all answers in this booklet.
      \item No calculators or notes are allowed.
      \item Best of luck!
    \end{itemize}
 \setlength{\leftskip}{0pt}

 \vfill
 \begin{center}\gradetable[h][questions]\end{center}

\newpage


\begin{questions}

\label{first_page}

\question[1] Under which of the following circumstance, we have $f$ grows faster than $g$?
\vspace{10pt}
\begin{choices}
    \choice $f(n) > g(n)$ for all natural number $n \in \mathbb{N}$.
    \choice $f(n) = C g(n)$, where $C$ is a constant coefficient, e.g., $C = 3$.
    \mycorrectchoice{$f(n) = n g(n)$}
    \choice $f(n) = - |g(n)|$
\end{choices}
\begin{solution}
    C.

    \Exp{
        \begin{align}
            & \lim_{n \to \infty} \frac{|Cg(n)|}{|g(n)|} = |C| \neq \infty \\
            & \lim_{n \to \infty} \frac{|n g(n)|}{|g(n)|} = \lim_{n \to \infty} |n| =  \infty \\
            & \lim_{n \to \infty} \frac{|-|g(n)||}{|g(n)|} = 1 \neq \infty
        \end{align}
    where $f(n) = Cg(n)$ for nonnegative $C$, $f$ and $g$ is a case of $f(n) > g(n)$, but it does not necessarily result in a faster $f$.
     }
\end{solution}

\vspace{20pt}

\question[1] Which of the following statements is true?\\

\begin{choices}
    \choice $\displaystyle \lim_{n\to \infty}\dfrac{n^4+3n+2}{n^5+4n+2}=1$
    \choice $\displaystyle \lim_{n\to \infty}\dfrac{n!+n\log_5(n)+10^{100}}{10000^{n}+n^5+1}=0$
    \mycorrectchoice{The function $f(n)=\log_2(n) + 2^n$ grows faster than the function $g(n)=n^2$}
    \choice The function $f(n)=n^5\log_2(n)+7$ grows faster than the function $g(n)=n^6+1.$
\end{choices}

\vspace{20pt}

\question[1] Evaluate the following limit: $\lim\limits_{n\to\infty} \frac{1}{3+e^{\frac{1}{n}}}$
\vspace{10pt}
\begin{oneparchoices}
    \choice $1$
    \choice $\frac{1}{3}$
    \mycorrectchoice{$\frac{1}{4}$}
    \choice $0$
\end{oneparchoices}
\begin{solution}
    C.
    
    \Exp{$\lim\limits_{n\to\infty} e^{\frac{1}{n}} = e^{0} = 1$. So, $\lim\limits_{n\to\infty} \frac{1}{3+e^{\frac{1}{n}}} = \frac{1}{4}$}
\end{solution}

\vspace{20pt}

\question Suppose that you are proving the statement $P(n)$ below by induction. 

\begin{center}
    \fbox{$P(n)$ = ``$9^n-1$ is divisible by $8$ for all natural number $n \in \mathbb{N}$".}
\end{center}

\vspace{10pt}

\begin{parts}
    \part[1] What should you prove in the base case?

    \begin{choices}
        \mycorrectchoice{$9^1-1$ is divisible by $8$.}
        \choice $9^0-1$ is divisible by $8$.
        \choice $9^{n}-1$ is divisible by $8$ for all $n \in \mathbb{N}$.
        \choice $9^{n-1}-1$ is divisible by $8$ for all $n \in \mathbb{N}$.
    \end{choices}
    
    \vspace{10pt}
    
    \part[1] What should you do in the induction step?
    
    \begin{choices}
        \choice Prove both $9^{n} - 1$ and $9^{n+1} - 1$ are divisible by $8$.
        \choice Enumerate $n=2,3,\dotsc$ and prove $9^{n} - 1$ is divisible by $8$.
        \choice Prove $9^{n} - 1$ is divisible by $8$ by assuming $9^{n+1} - 1$ is divisible by $8$.
        \mycorrectchoice{Prove $9^{n+1} - 1$ is divisible by $8$ by assuming $9^{n} - 1$ is divisible by $8$. }
    \end{choices}
\end{parts}

\vspace{20pt}

\question[5] For any natural number $n \in \mathbb{N}$, the factorial is $n! = 1 \cdot 2 \cdot \dotsc \cdot (n-1) \cdot n$.
\vspace{10pt}
\begin{parts}
    \part[3] Show that $n!$ can be computed by a recursion.
    \begin{solution}
        \begin{align}
             n!
             & = 1 \cdot 2 \cdot \dotsc \cdot (n-1) \cdot n \\
             & =  1! \cdot 2 \cdot \dotsc \cdot (n-1) \cdot n \\
             & = 2! \cdot \dotsc \cdot (n-1) \cdot n\\
             & = ...... \\
             & = (n-1)! \cdot n
        \end{align}
        Therefore, $n!$ can be computed by recursion.
        
        \MC{1.5 mark for showing equation $(n-1)! \cdot n$; 1.5 mark for all intermediate steps.}
    \end{solution}
    
    \vspace{10pt}
    
    \part[2] Write a recursive algorithm $\texttt{factorial}(n)$ that computes $n!$ for any input $n$.
    \begin{solution}
        
        \begin{algorithm}[H]
            \caption{\texttt{factorial}}
            \SetAlgoLined
            \SetKwInOut{Input}{input}\SetKwInOut{Output}{output}
            \SetKwFor{For}{for}{do}{endfor}
            \SetKwRepeat{Repeat}{repeat}{until}
            \SetKwIF{If}{ElseIf}{Else}{if}{then}{else if}{else}{endif}
            \BlankLine
            \Input{A natural number $n \in \mathbb{N}$.}
            \Output{$n!$}
            \BlankLine
            \lIf{n = 1}{output $1$}
            \lElse{output $n \cdot \texttt{factorial}(n-1)$}
        \end{algorithm}
        
        \MC{1 mark for base case; 1 mark for recursive case.}
    \end{solution}
\end{parts}

\label{last_page}

\end{questions}

\end{document} 